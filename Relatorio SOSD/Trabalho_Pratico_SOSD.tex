\documentclass{article}
\usepackage[T1]{fontenc}
\usepackage[utf8]{inputenc}
\usepackage[portuges]{babel}
\usepackage{graphicx}
\begin{document}
	\centerline{Instituto Politécnico Cávado do Ave \linebreak}
	\vspace*{1 em}
	\centerline{Curso engenharia de Sistemas Informáticos \linebreak}
	\vspace*{1 em}
	\centerline{Sistemas Operativos e Sistemas Distribuídos\linebreak}
	\vspace*{5 em}
	
	\centerline{Autores}
	
	\vspace*{3 em}
	\centerline{Carlos Santos Nº 19432\linebreak}
	\vspace*{1 em}
	\centerline{João Rodrigues Nº 19431\linebreak}
	\vspace*{1 em}
	\centerline{João Ricardo Nº 18845 \linebreak}
	\vspace*{3 em}
	
	\centerline{\textbf{Trabalho Prático}}
	
	\vspace*{10 em}
	\centerline{Data: 29/12/2020}
	
	\newpage
	%% ----------- Pagina 2
	\centerline{\textbf{ÍNDICE}}
	\vspace*{3 em}
	
	\begin{enumerate}
		\item Parte 1 \hspace{30 em} 3 \\
		\item Parte 2 \hspace{30 em} 5\\
		\item Parte 3 \hspace{30 em} 8\\
		\item Conclusão \hspace{28,5 em} 16 \\
		\item Execução das Tarefas \hspace{24 em} 17 \\
	\end{enumerate}
	%% ----------- Pagina 3
	
	\newpage
	\centerline{\textbf{Parte 1}}
	\vspace*{1 em}
	\centerline{Implementação de um conjunto de comandos para manipulação de ficheiros}
	\vspace*{3 em}
	
	Era pretendido que se implementa-se os seguintes comandos:
	\begin{enumerate}
		\item \textbf{mostra} ficheiro - Este comando deve mostrar no ecrã o conteúdo do ficheiro indicado. 
		\item \textbf{conta} ficheiro - Este comando deve contar o número de caracteres existentes de um ficheiro.
		\item \textbf{apaga} ficheiro - Este comando deve apagar o ficheiro com o nome indicado.
		\item \textbf{informa} ficheiro - Este comando apresenta a informação do sistema de ficheiros em relação ao ficheiro dado.
		\item \textbf{acrescenta} origem destino - Este comando deve acrescentar o conteúdo da "origem" no final do "destino".
		\item \textbf{lista} [caminho] - Este comando deve apresentar uma lista de todas as pastas e ficheiros existentes no caminho indicado ou na diretoria atual se não especificado.
	\end{enumerate}
	%% pagina 4
	\newpage
	\vspace*{2 em}
	Através de chamada de funções ao sistema(\textit{system calls}). Estes comandos implementados em C serão invocados através de um interpretador de comandos.
	
	\vspace*{4 em}
	\begin{figure}[!htb]
		\centering
		\includegraphics[scale=0.8]{apaga_parte1}
		\caption{Função Apaga}
		%%\label{Label de referência para a imagem}
	\end{figure}
	
	\vspace{2 em}
	Tanto nesta função como nas outras, o programa recebe como argumento o ficheiro ou diretoria em questão verificando sempre se realmente recebeu corretamente os argumentos requisitados. 
	De seguida executa a função system call e verifica com o valor devolvido pela função se ocorreu algum erro. Para finalizar o ficheiro deve ser fechado com a função close.
	
	\footnote[1]{https://linux.die.net/man/}
	
	\newpage
	
	% ------- Pagina 5 ------
	
	\centerline{\textbf{Parte 2}}
	\vspace*{1 em}
	\centerline{Implementação de um interpretador de comandos}
	\vspace{3 em}
	
	No sentido de substituir o interpretador de comandos habitual, por um novo interpretador personalizado, era importante implementar um programa que execute o comando através de primitivas de execução genérica de processos. \\
	Cada comando deverá dar origem a um novo processo e adicionalmente, poder considerar que a execução do interpretador deve ser suspensa até ao comando indicado estar concluído. O interpretador deverá indicar sempre se o comando conclui com ou sem sucesso, através do seu código de terminação/erro. O programa deverá permitir executar vários comandos sequencialmente, isto é, um a seguir ao outro, até	o utilizador indicar o comando especial “termina” que termina esta aplicação.
	
	\vspace*{4 em}

	\begin{figure}[!htb]
		\centering
		\includegraphics[scale=0.7]{interpretador_parte2}
		\caption{Funcionamento do interpretador}
		%%\label{Label de referência para a imagem}
	\end{figure}

	\newpage
	\begin{figure}[!htb]
		\centering
		\includegraphics[scale=0.7]{codigo_main}
		\caption{Código Main do Interpretador}
	\end{figure}

	\vspace{4 em}
	Na função main dentro de um while é esperado o input do utilizador para executar a função mysystem.
	
	\newpage
	
	\begin{figure}[!htb]
		\centering
		\includegraphics[scale=0.85]{codigo_mysystem}
		\caption{Código da Função mysystem}
	\end{figure}

	\vspace{4 em}
	
	Na função mysystem caso o input do utilizador não seja "termina" que encerra o programa,
	a função cria um novo processo filho que trata de executar o comando solicitado através da função execv. Entretanto, o processo-pai mantém-se à espera que o processo-filho acabe
	e devolva o código de terminação com que finalizou para poder informar o utilizador se ocorreu algum erro.

	\newpage
	
	% ------- Pagina 6 -------
	\centerline{\textbf{Parte 3}}
	\vspace{1 em}
	\centerline{Análise de cópia de ficheiros entre máquinas virtuais}
	\vspace{3 em}
	
	
	a) Configure a sua máquina virtual, de modo a que consiga comunicar com o host
	físico (máquina real).
	\vspace{2 em}
	\begin{figure}[!htb]
		\centering
		\includegraphics[scale=0.7]{tp_sosd_a1}
		\caption{Configuração da máquina Virtual para "BRIDGED ADAPTER"}
	\end{figure}
	
	\vspace*{2 em}
	Para que a máquina virtual conseguisse comunicar com o host foi necessário configurar a rede da máquina virtual para "Bridged Adapter".
	Assim, foi possível realizar o comando ping da máquina virtual para o host e do host para a máquina virtual com sucesso.
	Com o comando "ip address show" na máquina virtual e "ipconfig" no host foi possível confirmar o ip de ambas as máquinas.
	
	\newpage 
	\begin{figure}[!htb]
		\centering
		\includegraphics[scale=0.5]{tp_sosd_a2.1}
		\caption{Execução do comando ping - \textit{LINUX}}
	\end{figure}

	\vspace{2 em}

	\begin{figure}[!htb]
		\centering
		\includegraphics[scale=0.5]{tp_sosd_a2.2}
		\caption{Execução do comando ping - \textit{WINDOWS}}
	\end{figure}
	
	\newpage
	b) Recorrendo ao comando iperf3 mostre as diferenças de transferências entre a
	máquina real e virtual, usando tcp e udp.
	
	\begin{figure}[!htb]
		\centering
		\includegraphics[scale=0.5]{tp_sosd_b1.1}
		\caption{Execução do comando iperf - \textit{CLIENTE}}
	\end{figure}
	
	Recorreu-se ao comando iperf3 para mostrar as diferenças de transferência entre a máquina virtual e a real  
	Após 10 segundos as informações que são mostradas, na imagem acima. Neste exemplo, em 10 segundos foram transferidos 113 MBytes, atingindo a velocidade de média de 94,9 Mbits/sec. Conclui-se também que a taxa de envio foi superior a taxa de recepção.

	\vspace{1 em}
	\begin{figure}[!htb]
		\centering
		\includegraphics[scale=0.5]{tp_sosd_b1.2}
		\caption{Execução do comando iperf - \textit{SERVIDOR}}
	\end{figure}

	\footnote[1]{iperf3 -c , iperf3 -s  - Inicia o Iperf como Cliente e Servidor, respetivamente}
	
	\newpage
	c) Instale um servidor web (nginx ou apache) e mostre os resultados de um teste
	de carga ao “url /”, simulando 10 clientes em simultâneo e 500 pedidos cada
	um, utilizando o utilitário ab ou jmeter, sendo que o jmeter terá um acréscimo
	de 0.3 valores.
	\vspace{2 em}
	
	\begin{figure}[!htb]
		\centering
		\includegraphics[scale=0.6]{tp_sosd_c1}
		\caption{Servidor Web Apache e Utilitário JMeter}
	\end{figure}

	\newpage
	
	\begin{figure}[!htb]
		\centering
		\includegraphics[scale=0.6]{tp_sosd_c2}
		\caption{Configurações efetuadas ao JMeter}
	\end{figure}
	\newpage
	
	\begin{figure}[!htb]
	\centering
	\includegraphics[scale=0.5]{tp_sosd_c3}
	\caption{Resultados apresentados em Tabela}
	\end{figure}

	\newpage
	
	\begin{figure}[!htb]
		\centering
		\includegraphics[scale=0.5]{tp_sosd_c4}
		\caption{Resultados apresentados em Árvore}
	\end{figure}
	
	\vspace{3 em}
	
	Depois de instalado o servidor web apache foi realizado um teste de carga através do programa apache jmeter. Primeiramente foi configurado o "thread group" com 10 clientes e cada um com 500 pedidos. De seguido foi indicado qual o ip do servidor ao qual estava a ser feito o teste e o seu url.
	No fim foi possível verificar os resultados em tabela ou em árvore. 
	
	\vspace{1 em}
	Em tabela, é possível ver a lista de resultados e analisar o tempo, os bytes e a latência de cada "request". 
	
	\vspace{1 em}
	Em árvore podemos ver os resultados de forma mais detalhada.
	Todos os requests tiveram uma latência mínima e um número de bytes transferidos constante.

	\newpage
	d) Tendo como base o problema de copiar um ficheiro da primeira máquina
	virtual para a segunda máquina virtual (este ficheiro deverá ser criado na
	directoria /home/<nomeutilizador>/, com o nome sosd.txt e com o conteúdo
	“este ficheiro é para copiar”), indique os comandos que permitem realizar esta
	operação.
	Faça uso das seguintes sugestões de comandos para apresentar até 3 possíveis
	soluções distintas:
	1) scp
	2) dd, nc, pipe (|)
	3) cat, ssh, pipe (|)
	Apresente o comando para cada uma das três possíveis soluções com uma
	descrição de cada uma delas, indicando qual a melhor em termos de utilização
	de recursos e rapidez.
	vspace{2 em}
	
	\begin{figure}[!htb]
		\centering
		\includegraphics[scale=0.4]{tp_sosd_d1}
	\end{figure}
	
	\vspace{1,5 em}
	
	\begin{figure}[!htb]
		\centering
		\includegraphics[scale=0.5]{tp_sosd_d2}
	\end{figure}

	\vspace{1,5 em}
	
	Com o objetivo de copiar ficheiros entre máquinas foi necessário instalar o scp-server na máquina origem, na máquina destino, através do scp-client e do comando
	(scp [utilizador]@[ip\_origem]:[caminho\_ficheiro] [destino\_ficheiro]), foi copiado o ficheiro sendo assim possível obter uma cópia do ficheiro na máquina destino.


	%% Pagina 7
	\newpage
	\centerline{\textbf{Conclusão}}
	\vspace{5 em}
	
	Na nossa opinião foi muito interessante o desenvolvimento deste projeto, pois potenciou a experiência do desenvolvimento de Software. Assimilar os conteúdos da Unidade Curricular, desenvolver Capacidades de programação em \textit{C}\\
	
	\vspace{1 em}
	Sentimos que este projeto foi bastante exigente e fez com que nos dedicássemos mais e consequentemente melhorar as nossas capacidades.
	\\
	
	\vspace{1 em}
	Com este Trabalhos adquirimos inúmeras valias que nos serão úteis em futuros projetos.
	\\
	
	\vspace{1 em}
	Em suma, abordamos todos os assuntos lecionados e graças a isso conseguimos cumprir os objetivos propostos.
	\\
	
	% ------ Pagina 8
	\newpage

	\centerline{\textbf{Execução das Tarefas}}
	
	\vspace{3 em}
	\begin{enumerate}
		\item Parte 1 \\
			a) João Ricardo Nº 18845 \\
			b) João Ricardo Nº 18845 \\ 
			c) Carlos Santos Nº 19432 \\
			d) João Rodrigues Nº 19431 \\
			e) João Ricardo Nº 18845  \\
			f) João Rodrigues Nº 19431 \\
		\item Parte 2 \\ Carlos Santos Nº 19432 \\ João Ricardo Nº 18845 
		\item Parte 3 \\ Carlos Santos Nº 19432
	\end{enumerate}
	\vspace{3 em}
	
\end{document}